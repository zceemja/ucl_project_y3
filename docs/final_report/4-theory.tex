\iffalse
This chapter presents the background physical or electrical theory
and on any analytical methods you will use to accomplish your goals.
If you have a research question, what is it?
Have you made any deductions from it that you are now testing?
What mathematical bases must be understood in order to interpret your results in Chapter 5?
Give the reader a solid understanding of the foundations here.
\fi

\iffalse
Figure \ref{fig:simple_blocks} represents simplified diagrams of RISC and OISC architectures. In RISC and CISC architecture, program data travels from program memory to the control block where instruction is decoded. Then control block further decides how data is directed in the datapath block which is described in section \ref{sec:datapath}. Such structure requires a complicated control block and additional data routing blocks. In order to increase performance of one such processor you would need to add pipelining or multiple cores. Both methods have disadvantages: multicore processor requires software adjustments and each core doubles the control and datapath blocks, substantially increasing transistor count; pipelinig allows operation at higher frequencies however it brings design complications such as complicated hazard prevention logic and instruction lookup. RISC architecture in this project is mainly based on theory in \autocite{harris_harris_2013}. The simplicity of OISC architecture overcomes these disadvantages:

Pipelining can be done by individual blocks and programmibly waiting for results, this is represented in figure \ref{fig:oisc_simple} Adder and Multiply vertical blocks, multicore can be simulated by adding more data and instruction buses, hazards can be prevented with software and/or integrated into address registers.
\\ 
ALU and other processor components can be divided by adding different address registers. This allow utilisation of multiple components at the same time given that multiple data buses are used. This is represented in figure \ref{fig:oisc_simple} Arithmetic Unit horizontal blocks. Assuming 4 data and instructions buses are used, \textbf{AND} and \textbf{OR} blocks sources A and B can all be written during one cycle utilising both blocks at the same time.
\\
These 
\\
\fi

RISC that this paper will be exploring is classical SISO (single instruction, single operation) processor. TTAs are usually of type SIMT (single instruction, multiple transports) \autocite{289981}; A middle between these two classes is SIMO type (single instruction, multiple operation)

\textbf{Decided design criteria:}
\begin{description}
	\item[$\bullet$] Minimal instruction size
	\item[$\bullet$] Minimalistic design
\end{description}

%There are many papers looking into application specific TTAs. 