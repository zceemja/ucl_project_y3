

\subsection{Processor instruction set tables}\label{subsec:instruction_sets}
\arrayrulecolor{black}
\begin{longtable}[h!]{| l | p{.70\textwidth} | c |}
\caption{Instruction set for RISC processor. * Required immediate size in bytes}
\label{table:risc_instructions}\\

\hline
\rowcolor[rgb]{0.82,0.82,0.82}
Instr. & Description & I-size *\\\hline
\endhead		

\arrayrulecolor{black}\hline
\endfoot

\multicolumn{3}{|c|}{
	\cellcolor[rgb]{0.7,0.7,1}\textit{2 register instructions}} \\\hline
\arrayrulecolor[rgb]{0.82,0.82,0.82}

MOVE & Copy value from one register to other & 0 \\\hline
ADD  & Arithmetical addition & 0 \\
SUB  & Arithmetical subtraction & 0  \\
AND  & Logical AND & 0 \\
OR   & Logical OR & 0 \\
XOR  & Logical XOR & 0 \\
MUL  & Arithmetical multiplication & 0 \\
DIV  & Arithmetical division (inc. modulus) & 0 \\


\arrayrulecolor{black}\hline
\multicolumn{3}{|c|}{
	\cellcolor[rgb]{0.7,0.7,1}\textit{1 register instructions}} \\
\hline\arrayrulecolor[rgb]{0.82,0.82,0.82}


COPY0 & Copy intimidate to a register 0 & 1 \\
COPY1 & Copy intimidate to a register 1 & 1 \\
COPY2 & Copy intimidate to a register 2 & 1 \\
COPY3 & Copy intimidate to a register 3 & 1 \\\hline

ADDC & Arithmetical addition with carry bit& 0 \\
ADDI & Arithmetical addition with immediate & 1 \\
SUBC & Arithmetical subtraction with carry bit & 0 \\
SUBI & Arithmetical subtraction with immediate & 1 \\\hline

ANDI & Logical AND with immediate & 1 \\
ORI  &  Logical OR with immediate & 1 \\
XORI &  Logical XOR with immediate & 1 \\\hline

CI0  & Replace intimidate value byte 0 for next instruction & 1 \\
CI1  & Replace intimidate value byte 1 for next instruction & 1 \\
CI2  & Replace intimidate value byte 2 for next instruction & 1 \\\hline

SLL  & Shift left logical & 1 \\
SRL  & Shift right logical & 1 \\
SRA  & Shift right arithmetical & 1 \\\hline

LWHI & Load word (high byte) & 3 \\
SWHI & Store word (high byte, reg. only) & 0 \\
LWLO & Load word (low byte) & 3 \\
SWLO & Store word (low byte, stores high byte reg.) & 3 \\\hline

INC  & Increase by 1 & 0 \\
DEC  & Decrease by 1 & 0 \\
GETAH& Get ALU high byte reg. (only for MUL \& DIV \& ROL \& ROR) & 0 \\
GETIF& Get interrupt flags & 0 \\\hline

PUSH & Push to stack & 0 \\
POP  & Pop from stack & 0 \\
COM  & Send/Receive to/from com. block & 1 \\\hline

BEQ  & Branch on equal & 3 \\
BGT  & Branch on greater than & 3 \\
BGE  & Branch on greater equal than & 3 \\
BZ   & Branch on zero & 2 \\

\arrayrulecolor{black}\hline
\multicolumn{3}{|c|}{
	\cellcolor[rgb]{0.7,0.7,1}\textit{0 register instructions}
} \\
\hline\arrayrulecolor[rgb]{0.82,0.82,0.82} 

CALL & Call function, put return to stack & 2 \\
RET  & Return from function & 0 \\
JUMP & Jump to address & 2 \\
RETI & Return from interrupt & 0 \\
INTRE& Set interrupt entry pointer & 2 \\\hline


\end{longtable}	

\arrayrulecolor{black}
\begin{longtable}[h!]{| l | p{0.8\textwidth} |}
	\caption{Instructions for OISC processor.}
	\label{tab:oisc_instructions}\\
	
	\hline 
	\rowcolor[rgb]{0.82,0.82,0.82}
	Name & Description \\\hline
	\endhead		
	
	\arrayrulecolor{black}\hline
	\endfoot
	
	\multicolumn{2}{|c|}{
		\cellcolor[rgb]{0.7,0.7,1}\textit{Destination Addresses}} \\\hline
	\arrayrulecolor[rgb]{0.82,0.82,0.82}
	
	ACC0 & Set ALU source A accumulator \\
	ACC1 & Set ALU source B accumulator \\\hline
	BR0  & Set Branch pointer register (low byte) \\
	BR1  & Set Branch pointer register (high byte) \\
	BRZ  & If source value is 0, set program counter to branch pointer \\\hline
	STACK& Push value to stack \\
	MEM0 & Set Memory pointer register (low byte) \\
	MEM1 & Set Memory pointer register (middle byte) \\
	MEM2 & Set Memory pointer register (high byte) \\
	MEMHI& Save high byte to memory at memory pointer \\
	MEMLO& Save low byte to memory at memory pointer \\\hline
	COMA & Set communication block address register \\
	COMD & Send value to communication block \\\hline
	REG0 & Set general purpose register 0 \\
	REG1 & set general purpose register 1 \\
	
	\arrayrulecolor{black}\hline
	\multicolumn{2}{|c|}{
		\cellcolor[rgb]{0.7,0.7,1}\textit{Source Addresses}} \\\hline
	\arrayrulecolor[rgb]{0.82,0.82,0.82}
	
	NULL & Get constant 0 \\
	ALU0 & Get value at ALU source A accumulator \\
	ALU1 & Get value at ALU source B accumulator \\\hline
	
	ADD  & Get Arithmetical addition of ALU sources \\
	ADDC & Get Arithmetical addition carry \\
	ADC  & Get Arithmetical addition of ALU sources and carry \\\hline
	
	SUB  & Get Arithmetical subtraction of ALU sources \\
	SUBC & Get Arithmetical subtraction carry \\
	SBC  & Get Arithmetical subtraction of ALU sources and carry \\\hline
	
	AND  & Get Logical AND of ALU sources \\
	OR   & Get Logical OR of ALU sources \\
	XOR  & Get Logical XOR of ALU sources \\\hline
	
	SLL  & Get ALU source A shifted left by source B \\
	SRL  & Get ALU source A shifted right by source B \\
	ROL  & Get rolled off value from previous SLL instance \\
	ROR  & Get rolled off value from previous SRL instance \\\hline
	
	MULLO& Get Arithmetical multiplication of ALU sources (low byte) \\
	MULHI& Get Arithmetical multiplication of ALU sources (high byte) \\
	DIV  & Get Arithmetical division of ALU sources \\
	MOD  & Get Arithmetical modulus of ALU sources \\\hline
	
	EQ   & Check if ALU source A is equal to source B \\
	GT   & Check if ALU source A is greater than source B \\
	GE   & Check if ALU source A is greater or equal to source B \\
	NE   & Check if ALU source A is not equal to source B \\
	LT   & Check if ALU source A is less than source B \\
	LE   & Check if ALU source A is less or equal to to source B \\\hline
	
	BR0  & Get Branch pointer register value (low byte) \\
	BR1  & Get Branch pointer register value (high byte) \\	
	PC0  & Get Program counter value (low byte) \\
	PC1  & Get Program counter value (high byte) \\\hline
	
	MEM0 & Get Memory pointer register value (low byte) \\
	MEM1 & Get Memory pointer register value (middle byte) \\
	MEM2 & Get Memory pointer register value (high byte) \\
	MEMHI& Load high byte from memory at memory pointer \\
	MEMLO& Load low byte from memory at memory pointer \\\hline
	
	STACK& Pop value from stack \\
	ST0  & Get stack address value (low byte) \\
	ST1  & Get stack address value (high byte) \\
	
	COMA & Get communication block address register value \\
	COMD & Read value from communication block \\\hline
	
	REG0 & Get value from general purpose register 0 \\
	REG1 & Get value from general purpose register 1 \\
	
\end{longtable}	