% !TeX root = index.tex
\iffalse
The final chapter is short and sweet, to the point:
 what did you really accomplish? 
 What significant result can you claim and how it differs from anything done before. 
 How well did you meet your goal? 
 Now step out one level, then another indicating the impact your work will have on the literature and on future endeavours.
 * Start with the specifics and end with the general.
 * Summarise key result; mention limitations, note anything unexpected.
\fi

In this paper, two novel RISC and OISC-MOVE architectures are designed and implemented on FPGA. Logic element requirements, power consumption, maximum frequency where tested and timings of set of benchmark programs were executed to compare these two processors to investigate OISC-MOVE advantages. It is shown that power consumption differences are insignificant, RISC managed to reach 40\% higher maximum frequency at 75-70MHz, however due to a design fault in OISC. OISC required 51.7\% less logic elements to implement on FPGA. Benchmarks showed that OISC took 71\% longer to execute on average while requiring 41.71\% more instruction space. 

This project has sucessfully covered its goals in studying architectures and investigating alternative OISC implementation. Results shows that proposed implementation of OISC-MOVE may be only suitable for microprocessor application with very strict logic element limit. 

RISC processor has shown to be superior in tests, however it has more optimised implementation. Further research in needed to investigate OISC-MOVE performance with multiple data and instruction buses to match RISC complexity. 
