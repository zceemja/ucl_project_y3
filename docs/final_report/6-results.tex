\iffalse
This chapter looks specifically at your results.
* You measured some samples. 
What values did you measure? 
Present them in a table or graph? 
How did you test whether they were good measurements? 
Were you looking to improve something? 
Are your new samples better than the old ones?

* You built a device; 
what tests did you run to make sure that it is running correctly?

* You calculated something or developed a new theory about something. 
How do you know how well it predicts? 
What tests did you run? 
What comparisons with the literature did you make?
* You coded or simulated something. 
What tests did you run to be sure it was working correctly? 

Describe what you want the reader to notice in the results. 
Give the facts, then give your analysis of the facts.
Present your graphs, figures, tables, photos, and equations needed to show what you accomplished.
Label everything clearly, using the recommendations given below in “Things to Look For”
\fi

\subsection{Benchmark Programs}



\subsubsection{Number of instructions}

\subsubsection{Instruction composition}
Function composition was executed with following code:

\begin{lstlisting}[frame=single, caption={RISC assembly frame for executring tests}, emph={setup,start,done}]
setup:
	JUMP .start
.done:
	JUMP .done
.start:
	; Setup values
	; Call function
	JUMP .done
\end{lstlisting}

\begin{lstlisting}[frame=single, caption={OISC assembly frame for executring tests}, emph={setup,start,done}]
setup:
	BR1 .start @1
	BR0 .start @0
	BRZ 0x00
.done:
	BRZ 0x00
.start:
	; Setup values
	; Call function
	BR1 .done @1
	BR0 .done @0
	BRZ 0x00
\end{lstlisting}



\subsection{Maximum clock frequency}

\subsection{}
