\iffalse
The Introduction brings readers from a general understanding of the topic 
to a point where they can begin to understand what it is you are intending to do. 

It starts with broad statements and ends with specific statements about your project. 
Along the way it introduces readers to what has been done in the literature and 
then tells them why your project results will be different.

The Introduction provides a motivation for the work and tells readers what you will be telling them.
The following sections may be written simply as paragraphs; 
nothing more is really needed in the Introduction. 
You do not have to separate out each section.
The Funnel model is a good way to organise the Introduction.

1) The funnel model begins with a general statement about the general topical area;
 for example, “Antennas have been used for communications for at least 100 years”.
 It then narrows the focus repeatedly with further sentences by introducing work that has been 
 done in the literature with the appropriate citations.
 Finally, it reaches your project. By that time the reader knows in general terms what your work
 is about and understands your motivations. 
 The number of cited works ranges from a few to very many. 
 But whatever the number, they are the most significant in the field and have made the most impact on the historical development of the topic.

2) Your specific Aims and Objectives follow. Use bullet points for each and spend a few sentences describing each.

3) Follow your Aims and Objectives with a specific literature review. 
 In section 1, the review was rather broad. Now is the time to focus in on several journal articles 
 or products or activities that most closely match your own project work. 
 Use a few sentences to describe each one and show specifically what was useful about them. 
 Show how your work would improve on their work. You need only a few here. 
 Use those that are most similar and most like your project.

End the Introduction with a one-line description of the contents of each following Chapter.For example, “Chapter 2 focuses on.... Chapter 3 describes the work.... In Chapter 4, an outline of the measuring equipment ..., etc.”
\fi

Since the 70s there has been a rise of many processor architectures that try to fulfil specific performance and power application constraints. One of more noticeable cases are ARM's RISC  architecture being used in mobile devices instead of the more popular and robust x86 CISC (Complex Instruction Set Computer) architecture in favour of simplicity, cost and lower power consumption \autocite{jamil_1995,blem_menon_sankaralingam_2013}. It has been shown that in low power applications, such as IoTs (Internet of Things), OISC implementation can be superior in power and data throughput comparing to traditional RISC architectures \autocite{yokota_saso_hara-azumi_2017, ahmed_sakamoto_anderson_hara-azumi_2015}. This project proposes to compare two novel RISC and OISC 8bit architectures and compare their performance, design complexity and efficiency.


\subsection{Aims and Objectives}

The project has three main objectives:
\begin{enumerate}
	\item Design and build a RISC based processor. As this is aimed for low power and performance applications it will be 8bit word processor with four general purpose registers, structure is similar to MIPS.
	\item Design and build an OISC based processor. There are many different types of OISC processor, \texttt{MOVE} variant has been selected which is described in \refname{sec:theory} chapter. This type of OISC architecture variant may be also named TTA (Transport Triggered Architecture). 
	\item Design a fair benchmark that both processors could execute. This benchmark include different algorithms that are commonly used in controllers, IoT devices or similar low power microprocessor applications.
\end{enumerate}

\subsection{Supporting Theory}
This section goes though supporting theory of RISC and OISC architectures.

Principal functions of general OISC architecture should have advantage in performance and power consumption while having lower transistor count. This expectation is supported mainly by the following papers:\\
$\bullet$ Using OISC \texttt{SUBLEQ} as a coprocessor for the MIPS-ISA processor to emulate the functionality of different classes shows desirable area/performance/power trade-offs \autocite{ahmed_sakamoto_anderson_hara-azumi_2015}.\\
$\bullet$ Comparing OISC \texttt{SUBLEQ} multicore to RISC achieves better performance and lower energy for streaming data processing \autocite{yokota_saso_hara-azumi_2017}.
More specific OISC type - \texttt{MOVE} has been researched since early 90s. It showed that \texttt{MOVE} can benefits of VLIW (very large instruction word) arrangement, classifying it as SIMO (single instruction, multiple operation) or SIMT (single instruction, multiple transports) architectures. Problem with all of these arrangement is that they exhibit poor or complex hardware utilization. OISC \texttt{MOVE} has been proposed as a design framework enabling lower complexity, better hardware utilization, and scalable performance \autocite{5348869}. A MOVE32INT architecture as been designed \autocite{Corporaal94move32int} and proven to be superior architecture to RISC. Using 1.6$\mu m$ fabrication technology RISC achieved 20MHz clock with 20Mops/second, MOVE32INT implemented using SoGs (Sea of Gates) achieved 80MHz with 320Mops/second \autocite{289981}.

TTA framework as further used in other researches to implement Application-Specific Instruction Set Processors (ASIPs) to solve various problems. Some of the relevant examples are RSA calculation \autocite{6128530}; matrix inversion \autocite{1540373}; Fast Fourier Transform (FFT) \autocite{8682289}; IWEP, RC4 and 3DES encryption \autocite{922340}; Parallel Finite Impulse Response (FIR) filter \autocite{1511285}; Low-Density Parity-Check (LDPC) encoding \autocite{6855236}; Software Defined Radio (SDR) \autocite{7363689}. One of the most recent researches use TTA architecture to solve Compressive Sensing algorithms. It showed 9 times of energy efficiency that of FPGA implemented NIOS II processor, and theoretical 20 time energy efficiency that of ARM Cortex-A15 \autocite{8573494}. Most of these researches show that TTA has greater power efficiency, higher clock frequency, lower logic resource count. 

These benefits come with an expense, VLIW has bigger instruction word therefore bigger program size. TTA especially suffers from this due to redundant instructions. Some proposed solutions are variable length instructions with templates, which reduced program size between 30\% and 44\%; \autocite{1213033,6893206}; compression based on arithmetic coding \autocite{4627144}; and method to remove redundant instructions \autocite{5403730}. 
Software is another difficulty as compiler need to take additional steps for data transportation optimisations. TTA software can be easily exploited however, to embed software pipelining and parallelism without need of extra hardware\autocite{4595596}

With proposed \texttt{MOVE} framework hardware utilisation shown to be improved by reducing transition activity \autocite{1207041}, reducing interconnects shown saving 13\% of energy \autocite{6972455} on small scale. A novel architecture named SynZEN also showed a further improvements by using adaptable processing unit and simple control logic \autocite{6403142}.
\\[1cm]

Section \ref{sec:objectives} \ref{sec:theory} . Section \ref{sec:methods} explains both processor design choices and how each processor part is implemented on OISC and RISC processor. It also includes assembly design. In section \ref{sec:results}, results will be discussed, including benchmark methods. Summary and conclusion of design and results can be found in section \ref{sec:conclusion}. Appendix in section \ref{sec:appendix} includes any other information such as both processor instruction set.
% !TeX root = index.tex
\iffalse
The final chapter is short and sweet, to the point:
 what did you really accomplish? 
 What significant result can you claim and how it differs from anything done before. 
 How well did you meet your goal? 
 Now step out one level, then another indicating the impact your work will have on the literature and on future endeavours.
 * Start with the specifics and end with the general.
 * Summarise key result; mention limitations, note anything unexpected.
\fi

In this paper, two novel RISC and OISC-MOVE architectures were designed and implemented on a FPGA. Logic element requirements, power consumption, maximum frequency were tested. Benchmark programs execution times were used to compare these two processors and investigate OISC-MOVE advantages. It was shown that power consumption differences are insignificant, RISC managed to reach 40\% higher maximum frequency at 75-70MHz, however due to a timing design issue with OISC. OISC required 51.7\% less logic elements to implement on FPGA. Benchmarks showed that OISC took 71\% longer to execute on average while requiring 41.71\% more instruction space. 

This project has sucessfully covered its goals in studying architectures and investigating an alternative OISC implementation. Results show that proposed implementation of OISC-MOVE may be only suitable for microprocessor application with very strict logic element limit. 

RISC processor has been shown to be superior in tests, however it has more optimised implementation. Further research is needed to investigate OISC-MOVE performance with multiple data and instruction buses to match RISC complexity. 
