\iffalse
The Introduction brings readers from a general understanding of the topic 
to a point where they can begin to understand what it is you are intending to do. 

It starts with broad statements and ends with specific statements about your project. 
Along the way it introduces readers to what has been done in the literature and 
then tells them why your project results will be different.

The Introduction provides a motivation for the work and tells readers what you will be telling them.
The following sections may be written simply as paragraphs; 
nothing more is really needed in the Introduction. 
You do not have to separate out each section.
The Funnel model is a good way to organise the Introduction.

1) The funnel model begins with a general statement about the general topical area;
 for example, “Antennas have been used for communications for at least 100 years”.
 It then narrows the focus repeatedly with further sentences by introducing work that has been 
 done in the literature with the appropriate citations.
 Finally, it reaches your project. By that time the reader knows in general terms what your work
 is about and understands your motivations. 
 The number of cited works ranges from a few to very many. 
 But whatever the number, they are the most significant in the field and have made the most impact on the historical development of the topic.

2) Your specific Aims and Objectives follow. Use bullet points for each and spend a few sentences describing each.

3) Follow your Aims and Objectives with a specific literature review. 
 In section 1, the review was rather broad. Now is the time to focus in on several journal articles 
 or products or activities that most closely match your own project work. 
 Use a few sentences to describe each one and show specifically what was useful about them. 
 Show how your work would improve on their work. You need only a few here. 
 Use those that are most similar and most like your project.

End the Introduction with a one-line description of the contents of each following Chapter.For example, “Chapter 2 focuses on.... Chapter 3 describes the work.... In Chapter 4, an outline of the measuring equipment ..., etc.”
\fi
